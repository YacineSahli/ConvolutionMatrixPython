\documentclass[]{article}
% Use utf-8 encoding for foreign characters
\usepackage[utf8]{inputenc}


\usepackage{datetime2}
\usepackage{color}

\newcommand{\placeholder}[1]{{\noindent \color{red}[ #1 ]}}

\begin{document}

%\frontmatter          % for the preliminaries
%\pagestyle{headings}  % switches on printing of running heads
%\mainmatter              % start of the contributions

\title{
{\Huge Matrix Convolution Image Processing}\\
Multimedia Information System: Final Project Report\\}


\author{
\textbf{Sahli Yacine}\\
E10715005\\\
}


\date{{2018-2019}\\
\vspace{1cm}
National Taiwan University of Science and Technology(NTUST)}

\maketitle              % typeset the title of the contribution

\bigskip
\begin{center} \today \end{center}
\begin{abstract}
This \emph{implementation report} is handed for the course \textsf{MI5300701} "Multimedia Information Systems", given by the professor \emph{Yáng Chuán Kǎi} during the academic year  \emph{2018-2019} . \\    The goal of this report is to explain how my final project about Convolution Matrix Image Processing work. \\    I will start by explaining what is the goal of this project and the concept of convolution matrix image processing. \\    Then I will explain each functionality of my program and i will finish by enumerating what is yet to improve.
\end{abstract}

\newpage
\contentsline {section}{\numberline {1}Goals}{3}
\contentsline {section}{\numberline {2}What is a convolution Matrix(Kernel) ?}{4}
\contentsline {subsection}{\numberline {2.1}Edge handling}{4}
\contentsline {subsection}{\numberline {2.2}Normalization}{4}
\contentsline {section}{\numberline {3}How to use the program ?}{5}
\contentsline {section}{\numberline {4}What can be done ?}{6}
\contentsline {section}{\numberline {5}To be improved}{7}
\newpage
%%%%%%%%%%%%%%%%%%%%%%%%%%%%%%%%%%%%%%%%%%%%%%%%
%%%%%%%%%%%%%%%%%%%%%%%%%%%%%%%%%%%%%%%%%%%%%%%%


\section{Goals}
List of the goals I fixed myself with a grade on 5 according to the accomplishment of each one of them:
\begin {itemize}
	\item The program should be able to apply Convolution Matrix to images and save the modification in a new image. 5/5
	\item The user interface should be easy to use and simple. 4/5
	\item Anyone should be able to use this application. 3/5
	\item A set of comprehensible kernels should be available. 5/5
	\item At least 2 different ways to apply the kernel on the border's pixel should be given to the user. 5/5
	\item Possibility to transform the image in a gray-scale. 5/5
	\item It should work on Windows, MacOs and Linux. 5/5

\end {itemize}


\newpage
%%%%%%%%%%%%%%%%%%%%%%%%%%%%%%%%%%%%%%%%%%%%%%%%
%%%%%%%%%%%%%%%%%%%%%%%%%%%%%%%%%%%%%%%%%%%%%%%%
\section{What is a convolution Matrix(Kernel) ?}
In image processing, a convolution matrix or convolution kernel is a small matrix, usually 5x5 or 3x3, used for
image transformation such as embossing, blurring, sharpening, etc... \\
It works by applying a convolution between the kernel and an image.
\\\\
Convolution is the process of adding each element of the image to its local neighbors, weighted by the kernel.\\

\subsection{Edge Handling}

\begin{itemize}
	\item \textbf{Extend}\\
The value of the closest pixel is extended beyond the border.
	\item \textbf{Wrap}\\
The value is taken from the opposite border.
	\item \textbf{Mirror}\\
The value is taken from inside the picture as in a mirror, for example, if the pixel is 3 pixel out of the picture, we will take the
value of the pixel which is 3 pixel inside the picture from the border.
\end{itemize}


\subsection{Normalization}
Normalization consist in the division of each element in the kernel by the sum of all kernel elements,
so that the sum of the elements of a normalized kernel is one. We do this so that the average pixel in
the modified image is as bright as the average pixel in the original image.

%%%%%%%%%%%%%%%%%%%%%%%%%%%%%%%%%%%%%%%%%%%%%%%%
%%%%%%%%%%%%%%%%%%%%%%%%%%%%%%%%%%%%%%%%%%%%%%%%
\newpage
\section{How to use the program ?}
First, you should choose an image on your computer by clicking "Select an image" button, then you can select a convolution matrix preset 
in the "Presets" option bar at the top of the application's window.
\\\\
If you want, you can input you own convolution matrix manually in the "MATRIX" frame and input the according divisor 
necessary in the "Divisor" input box
\\\\
Then, you can check the "Grayscale" check-box if you would like your output image to be gray-scale.
\\\\
Choose how pixels at the border of the image should be handled in the "Border" frame by clicking on the according radio-box.
\\\\
You can now press preview to get a small, but very quickly computed, thumbnail of the output image.
\\\\
Before saving and starting the computation, you should choose whether you want the image to be scaled down to 1024 on its largest
dimension so that the computation is not too long.  \\The aspect ratio will be guaranteed !
\\\\
Then you can save the image and choose where you want to save it, the calculation will be done in a separate thread so that the graphical
user interface stays responsive.

\newpage

\section{What can be done ?}
A lot of things can be done with convolution Matrix, such as : 
\\
\begin {itemize}
	\item Blurring
	\item Sharpening
	\item Unsharpening
	\item Embossing
	\item Edge detection
	\item Edge enhancing
	\item Denoising
	\item It can help us understand convolution in general and apply it to other domain such as Machine Learning
\end {itemize}

The user is free to experiment his own Matrixes in the program. 
\\\\
\newpage

%%%%%%%%%%%%%%%%%%%%%%%%%%%%%%%%%%%%%%%%%%%%%%%%
%%%%%%%%%%%%%%%%%%%%%%%%%%%%%%%%%%%%%%%%%%%%%%%%


\section{To be improved}
\begin {itemize}
	\item Use numpy module to make the process faster as using python's list on large matrix is not very efficient.
	\item Rewrite from scratch in a lower level programming language like C the computing part for faster calculation by using python extension.
	\item Propose the user to apply the convolution kernel in a set of specifics color channels.
	\item Put the preview image directly into the application window instead of opening a new window.
	\item Let the user resize the image in his desired way.
	\item Make the user interface more user friendly and eye catching.
	\item Enabling the possibility to process multiple images at the same time in different threads.
	\item Add a cropping functionality
	\item Make the program compute the divisor by itself instead of asking for it as an input. ( but choosing the divisor can also be useful for the user )
	\item Implement a live preview, maybe with a refresh rate of 1 or 2 times every 5 seconds.
	\item Rewrite the user interface so that it looks better when resizing the window.
\end {itemize}

\end{document}
